% ====================== Pakiety =========================

% Support for UTF - relevant only for engines not supporting UTF
% \usepackage[utf8]{inputenc}

% Polski
\usepackage[polish]{babel}
\usepackage[T1]{fontenc}

% Marginesy
\usepackage[a4paper, top=2.5cm, bottom=3cm, inner=3cm, outer=2.5cm]{geometry}

% Stylizacja spisu treści
\usepackage{tocloft}

% Czcionki
\usepackage{fontspec}

% Nagłówki i stopki
\usepackage{fancyhdr}

% Odstępy między liniami
\usepackage{setspace}

% Bibliografia
\usepackage[backend=biber, sorting=none]{biblatex}
\usepackage{csquotes}

% Stylizacja nagłówków
\usepackage{titlesec}% Stylizacja nagłówków spisów
\usepackage{titletoc}

% Generowanie Lorem Ipsum
\usepackage{lipsum} 

% Wcinanie pierwszych akapitów
\usepackage{indentfirst}

% Hiperlinki
\usepackage[draft=false]{hyperref}

% Unikanie przełamania strony w akapicie
\usepackage[defaultlines=2,all]{nowidow}

% Zdjęcia
\usepackage{graphicx}
\usepackage{subcaption}

% Łączenie pdf-ów
\usepackage{pdfpages}

% Bloki komentarzy i bezpośredni tekst
\usepackage{verbatim}

% Stylizacja list
\usepackage{enumitem}

% Pakiet do wielu kolumn
\usepackage{multicol}

% Pakiet do oznaczania rzeczy do zrobienia
% \usepackage[bordercolor=white,backgroundcolor=gray!30,linecolor=black,colorinlistoftodos]{todonotes}

% Pakiet wspierający wiele opcjonalnych parametrów
\usepackage{xargs}

% Pakiet do umieszczania listingów kodu
\usepackage{listings}

% Kolory
\usepackage{color}

% Pakiet dotyczący języka francuskiego, ale usuwa sieroty
\usepackage[nosingleletter]{impnattypo}

% Poprawa typesettingu
\usepackage{microtype}

% ================== Konfiguracja =====================

% Ustawienie domyślnych czcionek
\setmainfont{Times New Roman}
\setsansfont{Arial}

% Odstęp między liniami
\setstretch{1.5}

% Głębokość wcięcie akapitów
\setlength\parindent{0cm}

% Usunięcie odstępu po tytule w spisie treści
\addtocontents{toc}{\vspace{-5ex}}

% Nagłówek strony
\pagestyle{fancy}
\fancyhf{}
\rhead{\fontsize{14pt}{1.2}\selectfont\textbf{\textit{\thepage}}}
\lhead{\fontsize{10pt}{12pt}\selectfont Łukasz Klimkiewicz\\ \textit{Wykorzystanie technologii mobilnych oraz Google Firebase na przykładzie portalu ogłoszeniowego}}

% Zapewnienie ilości miejsca na nagłówek
\setlength{\headheight}{22.5434pt}

% Dodanie nagłówka na każdej stronie
\fancypagestyle{plain}{}

% Usunięcie domyślnego tytułu spisu treści
\makeatletter
\renewcommand{\@cftmaketoctitle}{}
\makeatother

\makeatletter
\renewcommand{\@cftmakeloftitle}{}
\makeatother
  
% Zmniejsza automatycznie dodawane spacje między paragrafami
\setlength{\parskip}{3mm plus 1.0pt minus0mm}

% Usunięcie przerw przed rozdziałami i sekcjami
% Usunięcie przerw przez rozdziałami
\titlespacing{\chapter}{0pt}{-30pt}{0mm} % -\parskip
\titlespacing{\section}{0pt}{5mm}{0mm}
\titlespacing{\subsection}{0pt}{5mm}{0mm}

% Dodanie poprawnego numerowanie rozdziałów
\titleformat{\chapter}[hang] 
{\normalfont\LARGE\bfseries}{\thechapter}{1em}{} 

\titleformat*{\chapter}{\bfseries\LARGE\sffamily}
\titleformat*{\section}{\bfseries\Large\sffamily}
\titleformat*{\subsection}{\bfseries\large\sffamily}

% Dodanie bibliografii
\addbibresource{bibliography.bib}

% Konfiguracja hiperlinków
\hypersetup{
    colorlinks,
    citecolor=black,
    filecolor=black,
    linkcolor=black,
    urlcolor=black
}

% Ograniczenie głębokości spisu treści
\setcounter{tocdepth}{1}

% Usunięcie odstępu pomiędzy rozdział w spisie treści
\makeatletter
\renewcommand*\l@chapter[2]{%
  \ifnum \c@tocdepth >\z@
    \addpenalty\@secpenalty
    \setlength\@tempdima{1.5em}%
    \begingroup
      \parindent \z@ \rightskip \@pnumwidth
      \parfillskip -\@pnumwidth
      \leavevmode \bfseries
      \advance\leftskip\@tempdima
      \hskip -\leftskip
      #1\nobreak\hfil \nobreak\hb@xt@\@pnumwidth{\hss #2}\par
    \endgroup
  \fi}
\makeatother

% Usunięcie odstępu między elementami list i górnego odstępu
\setlist{nolistsep, topsep=0pt}

% Dostosowanie odstępu przed i po kolumnach
\setlength{\multicolsep}{3pt plus 0pt minus 0pt}

% Dodanie komendy do zakreślania
\newcommand{\stub}[1]{\todo[color=yellow,inline]{TODO: #1}}

% Komenda unsure
\newcommandx{\unsure}[2][1=]{\todo[linecolor=red,backgroundcolor=red!25,bordercolor=red,#1]{#2}}

% Odstęp pomiędzy ramką, a zawartością
\setlength{\fboxsep}{0pt}

% Definicja kolorów
\definecolor{lightgray}{rgb}{.95,.95,.92}
\definecolor{purple}{rgb}{1, 0, 1}
\definecolor{mediumorchid}{rgb}{0.72, 0.33, 0.82}
\definecolor{darkviolet}{rgb}{0.58, 0, 0.82}
\definecolor{green}{rgb}{0, 0.5, 0}

% Zdefiniowanie kolorowania dla typescript
\lstdefinelanguage{TypeScript}{
  keywords={typeof, new, true, false, catch, function, return, null, catch, switch, var, if, in, while, do, else, case, break, const, async, =>},
  keywordstyle=\color{blue}\bfseries,
  ndkeywords={class, export, boolean, throw, implements, import, this},
  ndkeywordstyle=\color{mediumorchid}\bfseries,
  identifierstyle=\color{black},
  sensitive=false,
  comment=[l]{//},
  morecomment=[s]{/*}{*/},
  commentstyle=\color{green}\ttfamily,
  stringstyle=\color{red}\ttfamily,
  morestring=[b]',
  morestring=[b]"
}

% Określenie stylów listingów
\lstset{
   language=TypeScript,
   backgroundcolor=\color{lightgray},
   extendedchars=true,
   basicstyle=\footnotesize\ttfamily,
   showstringspaces=false,
   showspaces=false,
   numbers=none,
   numberstyle=\footnotesize,
   numbersep=9pt,
   tabsize=2,
   breaklines=true,
   showtabs=false,
   captionpos=b
}

% Domyślna długość lorem ipsum
\SetLipsumDefault{2-6}

% Domyślnie wyśrodkowuj nagłówki zagnieżdżonych rysunków
\captionsetup[subfigure]{justification=centering}

% Wyłącza przełamywanie słów
% \tolerance=1
% \emergencystretch=\maxdimen
% \hyphenpenalty=10000
% \hbadness=10000

% ================== własne komendy =====================

\newcommand{\filename}[1]{\texttt{#1}}
\newcommand{\code}[1]{\texttt{#1}}