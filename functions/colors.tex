\section{Motywy kolorystyczne}

W celu odróżnienia od siebie obu aplikacji postanowiono zastosować dla nich różne motywy kolorystyczne. Mają one wpływ na wszystkie ekrany, które można zobaczyć w dalszej części rozdziału, jak i na ikony, które przedstawiono na rysunku \ref{fig:icons}.

\begin{figure}[ht!]
  \centering
  \fbox{\includegraphics[width=0.6\linewidth]{images/icons.png}}
  \caption{Ikony stworzonych aplikacji}
  \label{fig:icons}
\end{figure}

Motywy są stałe i nie można ich zmieniać, lecz poprzez to ułatwiają pracę użytkownikom, którzy posiadają zainstalowane obie aplikacje. Gdyby nie one, to ze względu na wiele podobieństw użytkownicy mogliby zostać zdezorientowani i mylić aplikacje w której aktualnie się znajdują.

Wybór koloru zielonego oraz niebieskiego jako motywów odpowiednio aplikacji dla klientów oraz wykonawców nie stanowi przypadku. Zgodnie z artykułem napisanym przez Hermana Cerato \cite{colors-meaning} każdy kolor ma inne znaczenie i zastosowanie. Skupia się on w szczególności na znaczeniu kolorów w domenie biznesowej. Według przytoczonego dokumentu kolor zielony jest kojarzony z harmonią, ma silny emocjonalny związek z poczuciem bezpieczeństwa i stabilności. Kolor niebieski za to najsilniej ze wszystkich jest wiązany z zaufaniem i niezawodnością. Wymienione pozytywne skojarzenia czynią wybrane barwy odpowiednimi do zastosowania w tworzonych aplikacjach.