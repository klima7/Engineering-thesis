\section{Praca offline}

Urządzenia mobilne, ze swojej natury, nie posiadają stabilnego połączenia z siecią. Z tego powodu tworzone aplikacje postanowiono wyposażyć w funkcjonalność pracy offline. Nie tworzono jej jednak z myślą o długotrwałym działaniu, lecz sporadycznych przypadkach. Postanowiono skupić się przy tym na możliwości odczytu informacji, natomiast brak możliwość ich modyfikacji uznano za akceptowalny przez użytkowników.

Duże ułatwienie stanowi wsparcie pracy offline przez bazę danych Firebase Firestore, która została wykorzystana przy większości operacji odczytu. Utrzymuje ona w pamięci cache swój lokalny obraz, by móc z niego korzystać, gdy połączenie z serwerem okaże się niemożliwe. Maksymalny rozmiar przetrzymywanych w tej pamięci informacji ustalono na 200 MB. Aby go utrzymywać najrzadziej wykorzystywane dane są usuwane. Oznacza to, że pracując bez połączenia z siecią należy się liczyć z brakiem kompletności dostępnych informacji. Baza umożliwia również przeprowadzanie modyfikacji danych, które ulegną synchronizacji w momencie przywrócenia połączenia. Zostało to wykorzystane w funkcji wysyłania wiadomości tekstowych na chacie.

% Można go ustalić na wartość dowolną, lecz wraz doskonaleniem tym sposobem pracy offline rośnie również sam rozmiar aplikacji, co nie jest korzystne z punktu widzenia użytkowników. Określając więc optymalny rozmiar pamięci cache należy znaleźć kompromis pomiędzy tymi aspektami, który mógłby być przedmiotem dalszych badań.

Magazyn plików Firebase Storage nie posiada żadnych mechanizmów ułatwiających pracę offline. W celu zapewnienia ciągłego dostępu do przechowywanych zdjęć, a przynajmniej ich części, wykorzystana została biblioteka Glide. Dostarcza ona wygodny oraz elastyczny mechanizm cachowania dla obrazów.

Duże trudności w zapewnieniu odpowiedniego działania pojawiły się tam, gdzie wykorzystane zostały funkcje Firebase Functions w celach odczytu informacji. Aby zapewnić pracę bez połączenia z siecią w tych przypadkach należało zaimplementować własny, zamknięty wewnątrz klasy odpowiedniego repozytorium, mechanizm cachowania.
