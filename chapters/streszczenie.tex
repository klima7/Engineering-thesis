% ==================== polski ==================== 
\section*{Streszczenie}
\vspace{1cm}

Celem niniejszej pracy dyplomowej było przedstawienie wykorzystania technologii mobilnych oraz platformy Google Firebase na przykładzie portalu pośredniczącego w świadczeniu usług. Miał on ułatwiać zamawianie ich przez klientów oraz oferowanie przez wykonawców.

Stworzone rozwiązanie umożliwia klientom dodawanie zleceń, które następnie są wysyłane do dopasowanych do nich wykonawców, mogących zgłaszać swoje oferty. Ze względu na dwie występujące w rozważanej domenie role zdecydowano się stworzyć również dwie aplikacje mobile. Jedna z nich przeznaczona jest dla usługobiorców, a druga dla usługodawców. Posiadają one chat zapewniający komunikację oraz system ocen i komentarzy, ułatwiający klientom podejmowanie właściwych wyborów. Przewidziano w nich również wsparcie dla pracy offline oraz dostosowanie interfejsu do ekranów tabletów. Platforma Firebase znacząco ułatwiła implementację powyższych elementów, lecz zauważono jej ograniczoną elastyczność.

% Celem niniejszej pracy dyplomowej było przedstawienie wykorzystania technologii mobilnych oraz platformy Google Firebase na przykładzie portalu pośredniczącego w świadczeniu usług. Miał on ułatwić łączenie klientów z wykonawcami oraz ich dalszą komunikację.

% Stworzone rozwiązanie polega na umożliwieniu klientom tworzenia zleceń, które następnie są wysyłane do odpowiednich wykonawców, mogących się do nich zgłaszać. Ze względu na dwie występujące w rozważanej domenie role zdecydowano się przygotować również dwie aplikacje mobilne. Jedna z nich jest przeznaczona dla usługobiorców, a druga dla usługodawców. Posiadają one chat, który zapewnia komunikację użytkowników oraz system ocen i komentarzy. Dzięki niemu klienci mogą wybierać wykonawcę usługi bazując na doświadczeniach wcześniejszych osób. W aplikacjach przewidziano również wsparcie dla pracy offline w ograniczonym zakresie oraz dostosowanie interfejsu do różnej wielkości ekranów. 


% Celem niniejszej pracy dyplomowej było przedstawienie wykorzystania technologii mobilnych oraz platformy Google Firebase na przykładzie portalu pośredniczącego w świadczeniu usług. Miał on umożliwiać klientom tworzenie zleceń, które następnie miały być wysyłane do odpowiednich wykonawców, mogących się do nich zgłosić. Na przygotowane rozwiązanie składają się dwie części: mobilne aplikacje klienckie oraz projekt Firebase.

% Ze względu na dwie występujące w rozważanej domenie role zdecydowano się stworzyć również dwie aplikacje mobilne. Jedna z nich jest przeznaczona dla usługobiorców, a druga dla usługodawców. Poza główną, opisaną funkcjonalnością, posiadają one chat, który zapewnia komunikację użytkowników oraz system ocen i komentarzy. Dzięki niemu klienci mogą wybierać wykonawcę usługi bazując na doświadczeniach wcześniejszych osób. W aplikacjach przewidziano również wsparcie dla pracy offline w ograniczonym zakresie oraz dostosowanie interfejsu do różnej wielkości ekranów. Dzięki temu wyglądają estetycznie zarówno na telefonach, jak i tabletach.

% Praca złożona jest z siedmiu rozdziałów poruszających następujące tematy. Pierwszy stanowi wstęp, w ramach którego przedstawiona została motywacja, cel i zakres pracy oraz założenia. Następy opisuje wykorzystane technologie, czyli wszystkie usługi, interfejsy, języki i środowiska, które odegrały rolę w procesie rozwoju projektu. Trzeci rozdział skupia się na bazie danych i porusza kwestię istniejącej w niej redundancji, a czwarty koncentruje na aplikacjach mobilnych. Została w nim dogłębnie opisana wykorzystana trójwarstwowa architektura oraz metoda współdzielenia kodu przez dwie tworzone aplikacje. W kolejnym rozdziale zawarto natomiast opis projektu Firebase i szczegółowo omówiono wszystkie jego części składowe, a w siódmym zostały obszernie omówione wszystkie zaimplementowane funkcjonalności. Ostatni rozdział stanowi podsumowanie całej pracy oraz proponuje możliwe drogi dalszego rozwoju aplikacji.

\vspace{1cm}
\noindent{\bf Słowa kluczowe:}
technologia mobilna, Android, Firebase, usługi
\clearpage

% ==================== angielski ==================== 
\section*{Abstract}
\vspace{1cm}

The aim of this thesis was to present the use of mobile technologies and the Google Firebase platform on the example of a service intermediary portal. It was supposed to facilitate ordering by customers and offering them by contractors.

The created solution allows customers to add orders, which are then sent to matching contractors, who can submit their offers. Due to the two roles occurring in the domain under consideration, it was decided to create also two mobile applications. One of them is intended for service recipients and the other for service providers. They have a chat feature for communication and a rating and commenting system to help customers make the right choices. They also provide support for offline work and adaptation of the interface to tablet screens. The Firebase platform has greatly facilitated the implementation of the above elements, but its limited flexibility has been noted.

% The aim of this thesis was to present the use of mobile technologies and the Google Firebase platform on the example of a service brokerage portal. It was supposed to facilitate connecting customers with contractors and their further communication.

% The created solution consists in enabling customers to create orders, which are then sent to the appropriate contractors, who may apply to them. Due to the two roles occurring in the domain under consideration, it was decided to also prepare two mobile applications. One of them is designed for service recipients and the other for service providers. They have a chat that provides communication between users and a rating and comment system. It allows customers to choose a service provider based on the experience of previous people. The applications also include support for limited offline work and adaptation of the interface to different screen sizes. 


% The aim of this thesis was to present the use of mobile technologies and the Google Firebase platform on the example of a service brokerage portal. It was supposed to enable customers to create orders, which were then sent to appropriate contractors, who could apply to them. The prepared solution consists of two parts: mobile client applications and Firebase project.

% Because of the two roles occurring in the considered domain, it was decided to create also two mobile applications. One of them is intended for service recipients and the other for service providers. Besides the main, described functionality, they have a chat, which provides communication between users, as well as a rating and comment system. It allows customers to choose a service provider based on the experience of previous customers. The applications also provide support for limited offline work and adaptation of the interface to different screen sizes. This makes them look aesthetically pleasing on both phones and tablets.

% The paper is composed of seven chapters covering the following topics. The first one is an introduction, which presents the motivation, goal, scope of the work and assumptions. The next describes the technologies used, that is all services, interfaces, languages and environments that played a role in the project development process. The third chapter focuses on the database and addresses the issue of redundancy in it, while the fourth focuses on mobile applications. It describes in depth the three-tier architecture used and the method of code sharing between the two applications being developed. The next chapter contains a description of the Firebase project and discusses all its components in detail, while the seventh provides an extensive description of all implemented functionalities. The last chapter is a summary of the whole work and suggests possible ways of further application development.

\vspace{1cm}
\noindent{\bf Keywords:} 
mobile technology, Android, Firebase, services
\clearpage
