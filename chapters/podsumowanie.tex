\chapter{Podsumowanie}

Podsumowując wielomiesięczne działania autora, założone w początkowej fazie projektu cele oraz funkcjonalności udało się zrealizować. Stworzone aplikacje mobilne ułatwiają klientom nabywanie usług, a wykonawcom ich świadczenie. Podczas rozwoju projektu zostało poruszonych wiele kwestii związanych zarówno z technologiami mobilnymi, jak i platformą Firebase.

Zgodnie z poczynionymi na wstępnie założeniami, aplikacje umożliwiają dodawanie zleceń przez klientów oraz zgłaszanie się do nich wykonawcom. Zawierają również chat usprawniający komunikację oraz system ocen i komentarzy. Poza wymienionymi funkcjonalnościami zaimplementowane zostało także wsparcie pracy offline, dostosowanie interfejsu do dużych ekranów oraz powiadomienia push.

Autor stwierdza, że bez wykorzystania platformy Google Firebase, lub jej podobnej, stworzenie tak zaawansowanego systemu przez jedną osobę wymagałoby znacznie większego wysiłku. Zauważa on jednak pewien brak jej elastyczności. Baza Firestore pozwala na wykonywanie jedynie prostych zapytań, a reguły bezpieczeństwa sprawiają trudności przy bardziej złożonych scenariuszach. Odpowiednia współpraca komponentów wymagała dogłębnej analizy na etapie projektowania. Może się to objawiać trudnościami w dodawaniu kolejnych funkcjonalności, które nie były przewidziane od samego początku.

% Autor stwierdza, że bez wykorzystania platformy Google Firebase, lub jej podobnej, stworzenie tak zaawansowanego systemu przez jedną osobę wymagałoby znacznie więcej wysiłku. Zauważa on jednak równocześnie, że bazowanie na gotowych komponentach znacząco zmniejsza elastyczność. Zapewnienie ich odpowiedniej współpracy w celu zaimplementowania pożądanych funkcjonalności wymagało dogłębnej analizy na etapie projektowania. Może się to powodować trudności w dodawaniu kolejnych funkcji, które nie były przewidziane od samego początku.

W ewentualnych, dalszych etapach rozwoju aplikacji, warto wprowadzić system umożliwiający weryfikację dodawanych ocen. Jego celem byłoby uniemożliwienie sztucznego podbijania ich przez nieuczciwych wykonawców. W celu nadania aplikacji wartości komercyjnej warto się również zastanowić nad systemem monetyzacji. Jest to szczególnie ważne dlatego, że Firebase jest płatnym rozwiązaniem, więc udostępnienie aplikacji w aktualnej wersji zapewne będzie generowało jedynie straty. 

% Wszystkie zadania zostały wykonane. Firebase umożliwia stworzenia w pełni funkcjonalnej aplikacji bez dodatkowej infrastruktury, zapewniając wszelkie niezbędne usługi. Minusem okazała się elastyczność. Baza Firestore jest ograniczona pod względem zapytań jakie może wykonać, Security Rules również nie umożliwiają walidacji we wszystkich scenariuszach. Zaimplementowanie wymaganych funkcjonalności wymagało dogłębnej analizy na etapie projektowania. Z tego powodu rozbudowywanie aplikacji o niezaplanowane wcześniej funkcjonalności, może okazać się kłopotliwe.

% \subsection{Możliwości dalszego rozwoju}
% \begin{itemize}
%   \item Funkcjonalności walidacji ocen dodawanych przez klientów, przez obsługę aplikacji, aby zapewnić ich wiarygodność
%   \item Dodanie monetyzacji - reklamy, bądź dodatkowe funkcjonalności po dokonaniu opłaty (możliwość wyboru przez eksperta większego obszaru działania, dłuższy czas aktywności zlecenia, większa liczba wykonawców mogąca się zgłosić)
%   \item Stworzenie panelu administracyjnego, umożliwiającego m.in. dodawanie nowych rodzajów usług i kategorii zamiast bezpośredniej manipulacji w bazie
%   \item Wsparcie dla usług zdalnych (takich gdzie wykonawcy nie muszą być w pobliżu)
% \end{itemize}