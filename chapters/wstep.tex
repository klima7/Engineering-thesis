\chapter{Wstęp}

Technologie mobilne cieszą się obecnie niewątpliwą popularnością. Telefony są wykorzystywane już nie tylko do komunikacji, lecz za sprawą różnorodnej funkcjonalności uczestniczą niemal we wszystkich dziedzinach życia. Według portalu Statista \cite{statista-mobiles-count} liczba aktualnie funkcjonujących urządzeń mobilnych wynosi 14,91 miliarda, a na najbliższe lata przewidywany jest dalszy wzrost ich liczebności. Lawinowo rośnie również liczba dostępnych aplikacji. Zgodnie z informacjami opublikowanymi przez ten sam portal \cite{statista-apps-count} w pierwszym kwartale 2021 roku w sklepie Google Play dostępnych było 3,48 miliona aplikacji, a w Apple App Store 2,22 miliona.

Sprostanie dużemu zapotrzebowaniu oraz wymaganiom gwałtownie zmieniającego się rynku wymaga szybkiego tworzenia i wprowadzania nowych aplikacji. Aby to ułatwić powstają i zyskują coraz większą popularność platformy dostarczające gotowych, kompleksowych rozwiązań w tym segmencie. Oferują szeroki wachlarz najczęściej wykorzystywanych usług, by deweloperzy nie byli zmuszeni do ich implementacji od podstaw, a mogli skupić na specyficznych dla aplikacji funkcjonalnościach. Wśród zapewnianych usług najczęściej znajdują się: baza danych, magazyn plików, system autoryzacji użytkowników, czy system notyfikacji push. Jedną z tego typu platform, cieszącą się powodzeniem, jest Google Firebase.

Zważając na dużą popularność aplikacji mobilnych oraz przydatne funkcjonalności oferowane przez platformę Google Firebase zdecydowano się na wykorzystanie oraz dokładniejsze zbadanie tych technologii.

% W niniejszej pracy postanowiono zbadać dokładniej możliwości oferowane przez technologie mobilne oraz Google Firebase.

\section{Cel oraz zakres pracy}

Celem pracy było przedstawienie wykorzystania technologii mobilnych oraz Google Firebase na przykładzie rozwiązania umożliwiającego łatwe zamawiania przez klientów i oferowanie przez wykonawców różnego rodzaju usług. 

Z góry został założony podstawowy schemat działania systemu polegający na tym, że utworzone przez klientów zlecenia na wykonanie usług miały być wysyłane do dopasowanych pod nie wykonawców - decydujących, czy chcą się ich podjąć. Jest to o tyle istotne, że jest to model odwrotny w stosunku do powszechnie znanego modelu sprzedaży dóbr materialnych, w którym to sprzedający tworzą oferty, a kupujący decydują, czy je przyjąć. W przypadku usług tę odpowiedzialność - podjęcie decyzji - warto przenieść na stronę wykonawców, ponieważ po zapoznaniu się z ofertą to oni wiedzą najlepiej, czy posiadają odpowiednie umiejętności i narzędzia, i czy mogą wykonać usługę w żądanym terminie. 

% W przedstawionym mechanizmie klientom pozostawia się swobodę wyboru, ponieważ, jeżeli do zlecenia zgłosi się kilku wykonawców, to mogą oni wybrać tego, który najbardziej im odpowiada. Mogą też nie wybrać żadnego, jeśli mają takie życzenie. Aby ten mechanizm sprawnie działał w zakres pracy zdecydowano się włączyć system ocen i komentarzy. Dzięki niemu klienci szukający wykonawców do nowo stworzonych zleceń będą mogli bazować swoje decyzje na ocenach i komentarzach dodanych przez innych.

W przedstawionym mechanizmie klientowi pozostawia się swobodę wyboru wykonawcy - jeśli do zlecenia zgłosi się ich kilku, wybiera tego, który najbardziej mu odpowiada. Ma również prawno nie wybrać żadnego, jeżeli stwierdzi, że nie spełniają oni jego oczekiwań. W celu dopełnienia tej funkcjonalności w zakres pracy został włączony system ocen i komentarzy. Dzięki niemu usługodawcy wybierający wykonawców do nowo utworzonych zleceń będą mieli możliwość podjęcia decyzji na podstawie ocen i komentarzy pozostawionych przez innych.

W zakres pracy włącza się chat, który ma służyć za podstawowy kanał komunikacji pomiędzy klientami i wykonawcami. Za jego pomocą mogą być uzgadniane szczegóły realizacji usług, czy też ich koszty. O pojawieniu się nowej wiadomości lub zajściu innego istotnego zdarzenia użytkownicy powinni być informowani za pomocą powiadomień push. Postanowiono również zadbać o przystosowanie aplikacji do tabletów oraz umożliwić użytkownikom - w ograniczonym zakresie - pracę offline.

\section{Założenia}

% Dwie aplikacje
W tworzonym rozwiązaniu wyraźnie widoczne są dwie role: klienta oraz wykonawcy. Nie wykluczają się one wzajemnie, ponieważ jeden użytkownik może posiadać je obie. Z tego powodu jednym z pierwszych założeń, jakie należało poczynić była decyzja, czy temat zostanie zrealizowany w postaci jednej, uniwersalnej aplikacji mobilnej, czy też dwóch aplikacji dedykowanych, z których jedna będzie przeznaczona dla klientów, a druga dla wykonawców. Ostatecznie zdecydowano się na realizację dwóch aplikacji, ponieważ spodziewano się, że zdecydowaną większość użytkowników będą stanowili klienci, dla których obecność dodatkowych funkcjonalności związanych z wykonawcami będzie zbędna. Z ich punktu widzenia funkcjonalności te będą dodatkowo komplikowały interfejs, zwiększały rozmiar aplikacji i mogą doprowadzić do zniechęcenia części z nich.

% Zamykanie zleceń
Postanowiono też określić, jak długo tworzone przez klientów zlecenia mają być aktywne, czyli kiedy możliwość zgłaszania przez wykonawców swoich ofert ma być blokowana. Uznano, że siedem dni to maksymalny okres, ponieważ nawet rzadziej korzystający z aplikacji wykonawcy powinni zdążyć w tym czasie zapoznać się ze zleceniem. Ustalono także limit na liczbę wykonawców, którzy będą mogli zgłosić się w ramach jednego zlecenia - na liczbę ośmiu. Oznacza to, że po zgłoszeniu się przez ósmego wykonawcę zlecenie będzie zamykane. Czyniąc w ten sposób wspierano się artykułem psychologicznym, poruszającym tematykę efektu zbyt dużego wyboru \cite{choice-complexity}. Stwierdza się w nim, że posiadanie zbyt wielu alternatyw ostatecznie prowadzi do negatywnych konsekwencji, takich jak obniżona satysfakcja dokonanym wyborem. Z tego względu przyjęcie wspomnianego limitu powinno zapewnić zadowolenie klientów, przy jednoczesnym zachowaniu wyboru, który wciąż wydaję się wystarczający.

% Dopasowywanie zleceń
Jednym z kluczowych aspektów jest sposób, w jaki wykonawcy mają być dopasowywani do zleceń. Założono, że aby wykonawca mógł się zgłosić do realizacji zlecenia, musi realizować żądaną usługę oraz wymagane miejsce realizacji musi znajdować się wewnątrz obszaru jego działania. Wykonawcy muszą więc określać realizowane przez siebie usługi oraz wspomniany obszar. Postanowiono, że będzie on kołem, ponieważ umożliwia to jego proste definiowanie. Wystarczą do tego jedynie dwa parametry: centralna lokalizacja oraz promień.

% Lokalizacja
Przed przystąpieniem do implementacji ważne było poczynienie założenia co do obszaru na którym zakłada się, że będzie ona używana. Ma to oczywisty wpływ na język, ale wewnątrz aplikacji przewidziana została również możliwość wybierania lokalizacji przez użytkowników, która powinna zostać ograniczona do zadanego rejonu. Musi określić ją chociażby klient dla każdego nowo tworzonego zlecenia, aby mogło zostać wysłane do pobliskich wykonawców. Brak jakichkolwiek ograniczeń oznaczałby, że osoby mieszkające w pobliżu granicy mogłyby otrzymywać i zgłaszać się do zleceń osób z sąsiednich krajów, co nie zawsze mogłoby być pożądane, a nawet możliwe. Pojawiają się wówczas również problemy natury komunikacyjnej z powodu prawdopodobnie różnych języków narodowych. Z tych względów stwierdzono, że taka aplikacja powinna być ograniczona do jednego kraju, a konkretnie postanowiono ją przygotować dla Polski.

% Optymalizacja kosztów
Ostatnim założeniem było zaimplementowanie wszystkich funkcjonalności, w miarę możliwości, w sposób zoptymalizowany pod względem kosztów. Większość usług wchodzących w skład platformy Firebase jest płatna, a obciążenia kosztami dokonuje się zwykle na podstawie intensywności ich wykorzystania. Zaniedbanie tego aspektu początkowo może nie stanowić problemu, lecz gdy aplikacja zyska popularność, to może ujawnić się brak jej skalowalności, objawiający się generowaniem wysokich kosztów. Z tego powodu należy przyjąć schematy przechowywania danych minimalizujące liczbę operacji, wykorzystać takie mechanizmy jak stronicowanie czy cachowanie, by temu zjawisku przeciwdziałać, a przynajmniej je ograniczać.
